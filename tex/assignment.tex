%%% generic article type (pdf)latex file
%%% use together with Makefile

\documentclass[letterpaper]{scrartcl}
\usepackage{graphicx}
\usepackage{amsmath,amsfonts,amsthm}
\usepackage{eufrak}
\usepackage{mathabx}
\usepackage{url}
\usepackage[usenames,dvipsnames,svgnames,table]{xcolor}
\usepackage[colorlinks]{hyperref}
\hypersetup{
     colorlinks   = true,
     urlcolor     = blue,
     linkcolor    = red,
     citecolor    = black
}
\usepackage{enumitem}
\usepackage{booktabs}
\usepackage{cprotect}
\usepackage{minted}


%\usepackage{wrapfig}
%\usepackage{subfig}
%\usepackage[format=plain,labelsep=period,font=small,labelfont=bf]{caption}

%------------------------------------------------------------
% assignment
%
\newcommand{\anumber}{4}
%
%------------------------------------------------------------

% hyperref https://en.wikibooks.org/wiki/LaTeX/Hyperlinks#.5Chref
\urlstyle{same}

%% not working yet...
\newcounter{TotalPoints}
\newcounter{TotalBonus}

\newcommand{\BONUS}{\textsc{Bonus: }}
\newcommand{\bonus}[1]{\textbf{[bonus +#1*]}\stepcounter{TotalBonus}}
\newcommand{\points}[1]{\textbf{[#1 points]}\stepcounter{TotalPoints}}
\newenvironment{enuma}{\begin{enumerate}[label=(\alph*)]}{\end{enumerate}}
\newenvironment{enumi}{\begin{enumerate}[label=(\roman*)]}{\end{enumerate}}
\newenvironment{solution}{\par\noindent\P{} }{\ \qedsymbol}

\renewcommand{\vec}[1]{\ensuremath{\mathbf{#1}}}
\newcommand{\pd}[3][]{\left(\frac{\partial #2}{\partial #3}\right)_{#1}}

\newcommand{\anum}{0\anumber}




\begin{document}
%\maketitle

\setcounter{section}{\anumber}
\addtocounter{section}{-1}
\section{ --- PHY 432: Classroom Activity (30 points total)}

\noindent For a refresher on using \textbf{NumPy} work through the
\href{https://numpy.org/doc/stable/user/quickstart.html}{NumPy
  tutorial}.\footnote{Some stuff such as the \texttt{ix\_()} function
  is fairly esoteric for beginners but almost everything else is what
  you should be familiar with for your daily work with arrays.} Do the
examples while you read it.

\noindent ``\BONUS'' problems are tested/autograded but you do not need to
solve the problem to get the maximum number of points.


\subsection{NumPy arrays (12 points)}


For the following, add your code to the file \texttt{problem1.py}.
Given the three arrays
\begin{minted}{python}
import numpy as np

sx = np.array([[0, 1], [1, 0]])
sy = np.array([[0, -1j],[1j, 0]])
sz = np.array([[1, 0], [0, -1]])
\end{minted}
\begin{enuma}
\item What is the result of \texttt{result1a = sx * sy * sz}?
  Explain what NumPy array multiplication does to the arrays. (Note:
  your code should assign the result to the variable
  \texttt{result1a} in \texttt{problem1.py}.) \points{2}
\item Use \texttt{np.dot()} to multiply the three arrays (like
  $\sigma_{x} \cdot \sigma_{y} \cdot \sigma_{z}$, where the dot
  ``$\cdot$'' indicates a matrix product). Add your code to
  \texttt{problem1.py} and assign your result to variable
  \texttt{result1b}. What happened? \points{2}
\item Compute the ``commutator''
  $[\sigma_{x}, \sigma_{y}] := \sigma_{x}\cdot\sigma_{y} -
  \sigma_{y} \cdot \sigma_{x}$ and show that it equals
  $2i\sigma_{z}$.\footnote{These are the Pauli matrices that
    describe the three components of the spin operator for a spin
    $1/2$ particle,
    $\hat{\vec{S}} = \frac{\hbar}{2}\boldsymbol{\sigma}$. The fact
    that any two components of the spin operator do \emph{not}
    commute is a fundamental aspect of quantum mechanics.} Add your
  code to \texttt{problem1.py}, assign the result to variable
  \texttt{result1c}. \points{3}
\item \BONUS Given a ``state vector''
  \begin{gather*}
    \vec{v} = \frac{1}{\sqrt{2}}\left(
      \begin{array}{c}
        1\\
        -i
      \end{array}\right)
  \end{gather*}
  calculate the ``expectation value''
  $\vec{v}^{\dagger} \cdot \sigma_{y} \cdot \vec{v}$ (i.e., the
  multiplication of the hermitian conjugate of the vector,
  $\vec{v}^{\dagger}$ with the matrix $\sigma_{y}$ and the vector
  $\vec{v}$ itself) using NumPy. \footnote{The \emph{hermitian
      conjugate} $\vec{v}^{\dagger} = (v_{1}^{*}, v_{2}^{*})$ is
    \texttt{v.conjugate().T} where \texttt{v.T} is shorthand for
    \texttt{v.transpose()}. It turns out that you don't need the
    transposition when you use \texttt{np.dot()} but I include it here
    for conceptual clarity. (Including \texttt{transpose()} comes at a
    minor performance penalty --- check with \texttt{\%timeit} if you
    are curious.)}  Add your code to \texttt{problem1.py} and assign
  your result to variable \texttt{result1d}.\footnote{Note for anyone
    having done PHY 315 (Quantum Mechanics II) that here you are
    calculating the quantum mechanical expectation value of the
    $y$-component of a spin $\frac{1}{2}$ particle in an eigenstate of
    the operator of the $y$-component of the spin
    ($\sigma_{y}\vec{v} = -\vec{v}$) and because $\vec{v}$ is
    normalized, you should get the eigenvalue as the expectation
    value.})  \bonus{2}
\end{enuma}


\subsection{Coordinate manipulation with NumPy (15 points)}
We can represent the cartesian coordinates
$\vec{r}_{i} = (x_{i}, y_{i}, z_{i})$ for four particles as a numpy array
\texttt{positions}:
\begin{minted}{python}
  import numpy as np
  positions = np.array(\
      [[0.0, 0.0, 0.0], [1.34234, 1.34234, 0.0], \
       [1.34234, 0.0,  1.34234], [0.0, 1.34234, 1.34234]])
  t = np.array([1.34234, -1.34234, -1.34234])
\end{minted}
and \texttt{t} will be a translation vector.
\emph{For the following use NumPy.} Add your code to file
\texttt{problem2.py} and assign results to variables as indicated in
the problems.
\begin{enuma}
\item What is the \emph{shape} of the array \texttt{positions} and
  what is its \emph{dimension}? Assign results to variables
  \texttt{results2a\_shape} and
  \texttt{results2a\_dimensions}. \points{2}
\item What is the \emph{shape} of the array \texttt{t} and what is its
  \emph{dimension}? Assign results to variables
  \texttt{results2b\_shape} and
  \texttt{results2b\_dimensions}. \points{2}
\item How do you access the coordinates of the second particle in
  \texttt{positions}? Assign the result to variable
  \texttt{result2c}. \points{1}
\item For the second particle:
  \begin{enumi}
  \item How do you access its $y$-coordinate? Assign the result to
    variable \texttt{result2d}.  \points{1}
  \item What type of object is this output, what is its \emph{shape}
    and its \emph{dimension}?\footnote{Do not assign results to any
      variables but answer the question for yourself.}
  \end{enumi}
\item How do you access the $y$-coordinates of \emph{all} particles?
  Assign to variable \texttt{result2e}. \points{1}

  What is the \emph{shape} (\texttt{result2e\_shape}) and number of
  (\texttt{result2e\_dimensions}) of the resulting array? \points{2}
  dimensions
\item \label{li:nptranslation} Write Python code to translate all
    particles by a vector $\vec{t} = (1.34234, -1.34234, -1.34234)$,
\begin{minted}{python}
t = np.array([1.34234, -1.34234, -1.34234])
\end{minted}
    Add your code to \texttt{problem2.py} and assign the translated
    coordinates to variable \texttt{result2f}. \points{3}
\item Make your solution of \ref{li:nptranslation} a function
  \texttt{translate(coordinates, t)}, which translates all
  coordinates in the argument \texttt{coordinates} (an
  \texttt{np.array} of shape \texttt{(N, 3)}) by the translation
  vector in \texttt{t}. The function should return the translated
  coordinates as a numpy array.

  Add the function to \texttt{problem2.py}. \points{3}

  Try out your function when applied to (1) the input
  \texttt{positions} and \texttt{t} from above and (2) for
  \texttt{positions2 = np.array([[1.5, -1.5, 3], [-1.5, -1.5, -3]])}
  and \texttt{t2 = np.array([-1.5, 1.5, 3])}.\footnote{Do not assign
    results to variables, just try out your \texttt{translate())}
    function on input where you can easily compute the \emph{expected
      result}.}
\end{enuma}


\subsection{NumPy function plotting with matplotlib (8 points)}
\label{sec:numpyfuncs}

We want to plot the function \footnote{This is the definition used in
  \href{https://numpy.org/doc/stable/reference/generated/numpy.sinc.html}{numpy.sinc}
  function.}
\begin{gather}
  \label{eq:sinc}
  \text{sinc}(x) := \frac{\sin(\pi x)}{\pi x}
\end{gather}
over the range $-6 \leq x < 6$.

\begin{enuma}
\item Use the NumPy
  \href{https://numpy.org/doc/stable/reference/generated/numpy.linspace.html}{\texttt{numpy.linspace()}}
  function to generate an array \texttt{X} with 601 values from -6 to
  6, including the upper boundary. \points{1}
\item Evaluate $\text{sinc}(x)$ (use the NumPy \texttt{sinc()}
  function) for all $x$ values in \texttt{X} \footnote{You do
    \emph{not} need any loops!}  and assign it to a variable
  \texttt{Y}. \points{4}
\item Plot your function with matplotlib:
  \begin{itemize}
  \item plot \texttt{X} vs \texttt{Y}
  \item label the $x$-axis with ``x'' and the $y$-axis with ``y = sinc(x)''
  \item write the plot to a PNG file \texttt{sinc.png}.
  \end{itemize}
  \points{3}
\end{enuma}
Submit your code as file \texttt{problem3.py} together with the plot
in file \texttt{sinc.png}.




%Total Points: \arabic{TotalPoints}. Total Bonus: \arabic{TotalBonus}*


\end{document}

%%% Local Variables: 
%%% mode: latex
%%% TeX-master: t
%%% End: 
